%% Generated by Sphinx.
\def\sphinxdocclass{report}
\documentclass[letterpaper,10pt,english]{sphinxmanual}
\ifdefined\pdfpxdimen
   \let\sphinxpxdimen\pdfpxdimen\else\newdimen\sphinxpxdimen
\fi \sphinxpxdimen=.75bp\relax

\PassOptionsToPackage{warn}{textcomp}
\usepackage[utf8]{inputenc}
\ifdefined\DeclareUnicodeCharacter
% support both utf8 and utf8x syntaxes
  \ifdefined\DeclareUnicodeCharacterAsOptional
    \def\sphinxDUC#1{\DeclareUnicodeCharacter{"#1}}
  \else
    \let\sphinxDUC\DeclareUnicodeCharacter
  \fi
  \sphinxDUC{00A0}{\nobreakspace}
  \sphinxDUC{2500}{\sphinxunichar{2500}}
  \sphinxDUC{2502}{\sphinxunichar{2502}}
  \sphinxDUC{2514}{\sphinxunichar{2514}}
  \sphinxDUC{251C}{\sphinxunichar{251C}}
  \sphinxDUC{2572}{\textbackslash}
\fi
\usepackage{cmap}
\usepackage[T1]{fontenc}
\usepackage{amsmath,amssymb,amstext}
\usepackage{babel}



\usepackage{times}
\expandafter\ifx\csname T@LGR\endcsname\relax
\else
% LGR was declared as font encoding
  \substitutefont{LGR}{\rmdefault}{cmr}
  \substitutefont{LGR}{\sfdefault}{cmss}
  \substitutefont{LGR}{\ttdefault}{cmtt}
\fi
\expandafter\ifx\csname T@X2\endcsname\relax
  \expandafter\ifx\csname T@T2A\endcsname\relax
  \else
  % T2A was declared as font encoding
    \substitutefont{T2A}{\rmdefault}{cmr}
    \substitutefont{T2A}{\sfdefault}{cmss}
    \substitutefont{T2A}{\ttdefault}{cmtt}
  \fi
\else
% X2 was declared as font encoding
  \substitutefont{X2}{\rmdefault}{cmr}
  \substitutefont{X2}{\sfdefault}{cmss}
  \substitutefont{X2}{\ttdefault}{cmtt}
\fi


\usepackage[Bjarne]{fncychap}
\usepackage{sphinx}

\fvset{fontsize=\small}
\usepackage{geometry}

% Include hyperref last.
\usepackage{hyperref}
% Fix anchor placement for figures with captions.
\usepackage{hypcap}% it must be loaded after hyperref.
% Set up styles of URL: it should be placed after hyperref.
\urlstyle{same}
\addto\captionsenglish{\renewcommand{\contentsname}{Contents:}}

\usepackage{sphinxmessages}
\setcounter{tocdepth}{1}



\title{EEP71}
\date{Oct 28, 2019}
\release{0.5}
\author{Dani Moeliker \& Damian Verbeek}
\newcommand{\sphinxlogo}{\vbox{}}
\renewcommand{\releasename}{Release}
\makeindex
\begin{document}

\pagestyle{empty}
\sphinxmaketitle
\pagestyle{plain}
\sphinxtableofcontents
\pagestyle{normal}
\phantomsection\label{\detokenize{index::doc}}


EEP71 project for the Raspberry Pi GUI.

Dependencies
—

opencv-python (cv2)
PyQt5 (REQUIRES INSTALLATION OF QT FRAMEWORK AS WELL (SEE README))


\chapter{EEP71: GUI}
\label{\detokenize{readme:eep71-gui}}\label{\detokenize{readme::doc}}

\section{Dependencies \& How to Install Them}
\label{\detokenize{readme:dependencies-how-to-install-them}}
This readme section is written with primarily Linux in mind.

1. python-opencv
Run \sphinxcode{\sphinxupquote{pip install opencv-python}}. Works the same way for Windows users.
If pip ran without error then you have successfully installed this dependency.

2. PyQt5
Run \sphinxcode{\sphinxupquote{pip install PyQt5}} to install the PyQt5 library. This still requires you to install Qt5.

For Arch users run \sphinxcode{\sphinxupquote{sudo pacman -S qt5-base}}.

For Ubuntu users see \sphinxurl{https://wiki.qt.io/Install\_Qt\_5\_on\_Ubuntu}.

For Windows users see \sphinxurl{https://www.qt.io/download-open-source\#section-2}.


\section{How to Run the Project}
\label{\detokenize{readme:how-to-run-the-project}}
Assuming all dependencies were installed correctly all you have to do is run \sphinxcode{\sphinxupquote{python mainwindow.py}} in your terminal emulator.


\chapter{EEP71: GUI Code (Autogenerated)}
\label{\detokenize{code:eep71-gui-code-autogenerated}}\label{\detokenize{code::doc}}\phantomsection\label{\detokenize{code:module-mainwindow}}\index{mainwindow (module)@\spxentry{mainwindow}\spxextra{module}}\index{UI\_Window (class in mainwindow)@\spxentry{UI\_Window}\spxextra{class in mainwindow}}

\begin{fulllineitems}
\phantomsection\label{\detokenize{code:mainwindow.UI_Window}}\pysigline{\sphinxbfcode{\sphinxupquote{class }}\sphinxcode{\sphinxupquote{mainwindow.}}\sphinxbfcode{\sphinxupquote{UI\_Window}}}~\index{nextFrameSlot() (mainwindow.UI\_Window method)@\spxentry{nextFrameSlot()}\spxextra{mainwindow.UI\_Window method}}

\begin{fulllineitems}
\phantomsection\label{\detokenize{code:mainwindow.UI_Window.nextFrameSlot}}\pysiglinewithargsret{\sphinxbfcode{\sphinxupquote{nextFrameSlot}}}{}{}
Reads the next frame from the OpenCV camera object and outputs it to a QPixmap.

\end{fulllineitems}

\index{openCamera() (mainwindow.UI\_Window method)@\spxentry{openCamera()}\spxextra{mainwindow.UI\_Window method}}

\begin{fulllineitems}
\phantomsection\label{\detokenize{code:mainwindow.UI_Window.openCamera}}\pysiglinewithargsret{\sphinxbfcode{\sphinxupquote{openCamera}}}{}{}
Opens the camera using OpenCV. This method works when only one camera is connected and will default to the first camera it can find.

\end{fulllineitems}

\index{stopCamera() (mainwindow.UI\_Window method)@\spxentry{stopCamera()}\spxextra{mainwindow.UI\_Window method}}

\begin{fulllineitems}
\phantomsection\label{\detokenize{code:mainwindow.UI_Window.stopCamera}}\pysiglinewithargsret{\sphinxbfcode{\sphinxupquote{stopCamera}}}{}{}
Stops the camera.

\end{fulllineitems}


\end{fulllineitems}



\chapter{Indices and tables}
\label{\detokenize{index:indices-and-tables}}\begin{itemize}
\item {} 
\DUrole{xref,std,std-ref}{genindex}

\item {} 
\DUrole{xref,std,std-ref}{modindex}

\item {} 
\DUrole{xref,std,std-ref}{search}

\end{itemize}


\renewcommand{\indexname}{Python Module Index}
\begin{sphinxtheindex}
\let\bigletter\sphinxstyleindexlettergroup
\bigletter{m}
\item\relax\sphinxstyleindexentry{mainwindow}\sphinxstyleindexpageref{code:\detokenize{module-mainwindow}}
\end{sphinxtheindex}

\renewcommand{\indexname}{Index}
\printindex
\end{document}